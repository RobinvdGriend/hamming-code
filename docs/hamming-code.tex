\documentclass[a4paper]{article}
\usepackage[dutch]{babel}
\usepackage{mathtools}

\begin{document}
\title{Hamming-code in Python 3.6}
\author{Robin van de Griend, Thomas Koopman, Tim Waroux}
\maketitle

\section{Taakverdeling}

\section{Complexiteit}
We drukken de complexiteit uit in de lengte van de boodschap \(n\). Of dit in bits of in tekens is, maakt niet uit omdat elk teken 8 bits is, en in grote O notatie maakt dit niet uit.

Alle functies behalve encodeentiremessage, repairentiremessage, destroyallparitybits en decodeentiremessage nemen niet de
boodschap als argument. Deze zijn dus O(1).

encodeentiremessage is O(n) want message is een lijst van n matrices (elk karakter is 1 matrix), en je loopt 1 keer door de lijst heen, en voert elke keer een constant aantal operaties uit. Een lege lijst new_message maken en dit returnen hangt niet van n af en is dus constant.

Het geval repairentiremessage is analoog.

Het geval destroyallparitybits is analoog.


\end{document}
