\documentclass[a4paper]{article}
\usepackage[dutch]{babel}
\usepackage{mathtools}

\begin{document}
\title{Hamming-code in Python 3.6}
\author{Robin van de Griend, Thomas Koopman, Tim Waroux}
\maketitle

\section{Taakverdeling}

\section{Ontwerp van de code}
We hebben er voor gekozen om gebruik te maken van verschillende modules, we hebben voor belangrijke functies een apart bestand aangemaakt. Dit is beter voor de leesbaarheid en maakt het makkelijker om te werken in de code. Hieronder volgt een toelichting voor de verschillende modules.

\subsection{matrix.py}
	Om te beginnen met het programmeren van de Hammingcode, hebben we ervoor gekozen om eerst een klasse Matrix te definieren. Dit staat eveneens aangegeven in de opdracht die we gevolgd hebben. Hier is al onze code opgebasseerd omdat we bij onze hammingcode gebruik maken van de matrixvermenigvuldiging.

\subsection{strconv.py}
	Onze stringconvert bestand is van essentieel belang voor onze hammingcode. De functies die gedefinieerd staan in stringconvert zorgen ervoor dat een string, met name de teskt die mensen willen versturen of opslaan, omzet naar een lijst met matrices. Hierdoor kunnen we ze gebruiken in het Hamming bestand, waar onze Hammingcode wordt uitgevoerd.

\subsection{hamming.py}
	
\section{Complexiteit}
We drukken de complexiteit uit in de lengte van de boodschap \(n\). Of dit in bits of in tekens is, maakt niet uit omdat elk teken 8 bits is, en in grote O notatie maakt dit niet uit.

Alle functies behalve encodeentiremessage, repairentiremessage, destroyallparitybits en decodeentiremessage nemen niet de
boodschap als argument. Deze zijn dus O(1).

encodeentiremessage is O(n) want message is een lijst van n matrices (elk karakter is 1 matrix), en je loopt 1 keer door de lijst heen, en voert elke keer een constant aantal operaties uit. Een lege lijst new\_message maken en dit returnen hangt niet van n af en is dus constant.

Het geval repairentiremessage is analoog.

Het geval destroyallparitybits is analoog.
\end{document}